\section{Problem Description}

To motivate the problem of estimating thermal coupling between microzones, we provide a brief overview of buildings and their control systems.
Then, we discuss the potential utility of statistical models over complex physical building models, and how we apply parameter estimation in an Ising model to a particular building modeling problem.

\subsection{Motivation}



\if 0
Why Buildings?
- buildings use 40% of energy:
    - have lots of sensors, automated control systems
    - these are often incomplete or out of date
    - much manual effort to repair this, or to build up metadata
    - metadata can help buildings be more efficient and comfortable:
        - simultaneous heating/cooling
        - misplaced thermostats (like behind 410 soda screen)
    - want to reason about how buildings behave:
      - building models one approach, but expensive and hard to do and become out of date
      - need another method, one that can be somewhat automated and not require
        immense amounts of domain expertise

Problem Description:
- buildings assume temperature is uniform across an HVAC zone: how true is this really?
    - two complementary questions:
      1. Can we determine the nature of thermal coupling between sub-hvac-zones?
      2. Could we reconstruct the assignment of rooms to hvac zones from this information?
    - We don't address the second, but we explore the first

\fi
